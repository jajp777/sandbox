\documentclass[10pt,a4paper]{book}
\usepackage[latin1]{inputenc}
\usepackage{amsmath}
\usepackage{amsfonts}
\usepackage{amssymb}
\usepackage{graphicx}
\author{Daniel Frederico Lins Leite}
\begin{document}

Aula 1

Amizade = idem velle idem nolle
S�o Tom�s

Arist�teles: a amizade � a base pol�tica.

Exerc�cio do Necrol�gio: a id�ia � encontrar a "voz" mais alta dentro de voc� mesmo que ser� aquela que ir� lhe julgar, lhe direcionar. Seria a "voz" da confiss�o Cat�lica.

filosofia como: unidade do conhecimento na unidade da consci�ncia

filosofia se constitui com afficcionados como Socrates, Platao e Aristoteles. Por�m na idade m�dia se profissionaliza, o que abre uma �tima oportunidade por�m vai aumenta a exig�ncia sociais dos fil�sofos profissionais que vai matando a filosofia em si. Pois o professor vai tendo que cuidar da faculdade, economicamente, administrativamente, socialmente etc... E de algum modo isso vai influenciar a proprie produ��o filos�fica, mesmo que os fil�sofos n�o percebam. Isso faz uma s�rie de filosofias que n�o s�o conscientes das suas condi��es sociais.
Um dos exemplos mais �bvios � que um professor universit�rio, dentro de um curso, normalmente j� est� limitado pela delimita��o do pr�prio curso, sem nunca se perguntar se a realidade pode ser descrita por esta limita��o.

Em Socr�tes a filosofia come�a como filosofia pol�tica. A situa��o social da Gr�cia e dos interlecutores.
\end{document}