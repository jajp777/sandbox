\documentclass[10pt,a4paper]{book}
\usepackage[latin1]{inputenc}
\usepackage{amsmath}
\usepackage{amsfonts}
\usepackage{amssymb}
\usepackage{graphicx}
\author{Daniel Frederico Lins Leite}
\title{Syllogisms}
\begin{document}
	
	\maketitle
	
	\section{TODO}
	
	TODO: Aristotle
	TODO: Theophrastus
	TODO: Eudemos
	
	\index{entry}
		
	\begin{enumerate}
		\item Categorical Syllogisms
		\item Hypothetical Syllogisms
	\end{enumerate}

	\chapter{Aristotle}
	\chapter{Theophrastus}
	\chapter{Alexander of Aphrodisias}
	\chapter{Ioannes Philoponos}
	\chapter{Eudemos}	
	\chapter{Boethius}	

	\chapter{References}
	
	from "The Propositional Logic of Boethius : Studies in Logic and the Foundations of Mathematics"\\ 
	page 4	\\
	Boethius speaks to a certain Symmachus who is named in the title of the work.\\ 
	"What I have found in a few Greek authors brief and confused and in no Latin authors, that I dedicate to your insight, after its having been completed by our long, but successful exertions. When you had acquired a comprehensive knowledge of categorical syllogisms, you often desired information about hypothetical syllogisms, concerning which Aristotle wrote nothing. Theophrastus, a man versed in every science, works out only the main points. Although Eudemos touches on the subjects more broadly, lie proceeds in such a fashion that he seems to have only thrown out a few seeds without having harvested any of the fruit."\\
	(Cf. Boe. p. 606). 
	\\
	"In his article "Zur Geschichte der Aussagenlogik", Jan Lukasiewicz opposed propositional logic to the logic of names (Cf. Lu. p. 111). If we adopt this usage, we can say: The theory of categorical 
	syllogisms is the logic of names and the theory of hypothetical syllogisms is propositional logic."\\
	\\
	page 5\\
	Boethius relates the theory of hypothetical syllogisms with Theophrastus and Eudemos; it is unmistakable that he regards these two peripatetics as the first to develop this kind of logic. 
	This last fact is significant, because according to the interpretation now represented by Jan Lukasiewicz and Heinrich Scholz, which we may call the modern interpretation, the Stoics either founded 
	propositional logic or at least developed it to a higher level (Cf. Lu. p. 112 and Sch. p. 31). This raises the question of the relation between Boethius and the Stoic logic.
	page 6\\
	In order to establish what the theory of the older peripatetics was, we are directed to the information to be found in Greek commentaries on Aristotle's Prior Analytics. Aristotle speaks, in several places, of hypothetical syllogisms. These remarks give the commentators occasion to point out what the doctrines of the older peripatetics on hypothetical syllogisms were. There are two commentaries to be taken into consideration, namely: 
	(1) The commentary of Alexander of Aphrodisias. 
	(2) The commentary of Ioannes Philoponos. 
	
	
	
\end{document}