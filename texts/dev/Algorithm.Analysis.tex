\documentclass{article}
\usepackage{listings}
\usepackage{dirtree}

\begin{document}
	
Recursive Call Count Operator

if f is a function, \#f returns how much recursive call f does.
	
Fibonnaci Numbers

\lstset{language=C}
\begin{lstlisting}[frame=single]
int fibonnaci(int n){
	if(n <= 2) return 1;
	return fibonnaci(n-1) + fibonnaci(n-2);
}
\end{lstlisting}

Call tree\\

\dirtree{%
	.1 fibonnaci(5).
	.2 fibonnaci(4).
	.3 fibonnaci(3).
	.4 fibonnaci(2) = 1.
	.4 fibonnaci(1) = 1.
	.3 fibonnaci(2) = 1.
	.2 fibonnaci(3).
	.3 fibonnaci(2) = 1.
	.3 fibonnaci(1) = 1.
}

All leafs always have value 1. So\\ 
fibonnaci(n) = 1 + 1 + ... + 1 + 1\\
In this case the call tree have:\\
9 nodes = \\
5 leafs\\
4 aggregators\\

Every aggregator comes from a recursive call. So\\
fibonnaci(5) makes 4 recursives calls.\\

So
\#fibonnaci(n) = \#fibonnaci(n-1) + \#fibonnaci(n-2) + 1\\

Thesis:\\
\#fibonnaci(n) = fibonnaci(n) - 1\\

Proof by Induction\\
Base\\
\#fibonnaci(1) = fibonnaci(1) - 1 = 1 - 1 = 0 (OK)\\
\#fibonnaci(2) = fibonnaci(2) - 1 = 1 - 1 = 0 (OK)\\

Step\\
\#fibonnaci(n) = fibonnaci(n) - 1\\
\#fibonnaci(n) + fibonnaci(n-1) = fibonnaci(n) - 1 + fibonnaci(n-1)\\
\#fibonnaci(n) + fibonnaci(n-1) = fibonnaci(n) + fibonnaci(n-1) - 1\\
\#fibonnaci(n) + \#fibonnaci(n-1) + 1 = fibonnaci(n+1) - 1\\
\#fibonnaci(n+1) = fibonnaci(n+1) - 1\\
QED


Binomial Numbers

\lstset{language=C}
\begin{lstlisting}[frame=single]
int binomial(int n, int k){
	if(k > n) return 0;
	if(k == 0 || n == k) return 1;
	return binomial(n-1,k) + binomial(n-1,k-1)
}
\end{lstlisting}

Call Tree:

\dirtree{%
	.1 binomial(4,2).
	.2 binomial(3,2).
	.3 binomial(2,2) = 1.
	.3 binomial(2,1).
	.4 binomial(1,1) = 1.
	.4 binomial(1,0) = 1.
	.2 binomial(3,1).
	.3 binomial(2,1).
	.4 binomial(1,1) = 1.
	.4 binomial(1,0) = 1.
	.3 binomial(2,0) = 1.
}

All leafs always have value 1. So\\ 
binomial(n,k) = 1 + 1 + ... + 1 + 1\\
In this case the call tree have:\\
11 nodes = \\
6 leafs\\
5 aggregators\\

Every aggregator comes from a recursive call. So\\
binomial(4,2) makes 5 recursives calls.\\

\#binomial(n,k) = \#binomial(n-1,k) + \#binomial(n-1,k-1) + 1\\

Thesis:\\
\#binomial(n,k) = binomial(n,k) - 1\\

Bases:\\
\#binomial(1,0) = binomial(1,0) - 1 = 1 - 1 = 0 (OK)\\

Step:\\
\#binomial(n,k) = binomial(n,k) - 1\\
\#binomial(n,k) + binomial(n,k+1) = binomial(n,k) - 1 + binomial(n,k+1)\\
\#binomial(n,k) + \#binomial(n,k+1) + 1 = binomial(n+1,k+1) - 1\\
\#binomial(n+1,k+1) = binomial(n+1,k+1) - 1\\
QED\\

\end{document}