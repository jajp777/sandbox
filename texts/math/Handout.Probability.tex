\documentclass[10pt,a4paper]{book}
\usepackage[latin1]{inputenc}
\usepackage{amsmath}
\usepackage{amsfonts}
\usepackage{amssymb}
\usepackage{graphicx}
\newcommand\tab[1][1cm]{\hspace*{#1}}
\author{Daniel Frederico Lins Leite}
\title{Probability}
\begin{document}
	\section{Introduction}
	\subsection{Subset}
	
	\subsection{Population}
	A \textbf{population} is defined as a collection of statistical units of the same nature whose quantifiable information we are interested in. The 	population constitutes the reference universe during the study of a given statistical problem.
	\subsection {Sample}
	A \textbf{sample} is a \textit{subset} of a \textit{population} on which statistical studies are made in order to draw conclusions relative to the \textit{population}.
	$w$ is a "sample point".\\
	$S$ is a "sample space" if\\
		\tab $w$ is a "sample point and $w \in S$;\\
		\tab and $\forall_{i,j}{(w_i \cap w_j = \emptyset \text{, if } i \neq j)}$;\\
		\tab and $w_1 \cup w_2 \cup ... \cup w_n = S$.\\
		A is a "family of events" if is a set of "sample points"\\
		\tab $A = \{w\}$ where $w$ is a "sample point"\\
		\tab $A^\complement = \{w : w \notin A\}$\\
		\tab $A \cup B = \{w: w \in A \lor w \in B\}$\\
		\tab $A \cap B = \{w: w \in A \land w \in B\}$\\
		\tab $S^\complement = \emptyset$\\
		\tab $A \cup A^\complement = S$\\
		\tab $A \cap A^\complement = \emptyset$\\
		\tab $A \cap S = A$\\
		\tab $A \cup S = S$\\
		\tab $A \cup \emptyset = A$\\
		\tab $\cup$ is commutative, associative, distributive\\
		\tab $\cap$ is commutative, associative, distributive\\
		P is a "probability measure" if is a mapping between S and the "real numbers" with the following properties:\\
		\tab $P = f: S \mapsto {\rm I\!R}$\\
		\tab $P(A) = f(A)$\\
		\tab $f(S) = \sum_{\forall i}{f(A_i)} = 1$\\
		\tab $0 <= f(A) <= 1$\\
		\tab if $A \cap B = \emptyset$ then $f(A \cup B) = f(A) + f(B).$\\
		\\
		The triplet (S,A,P) defines a "probability system", a consistent axiomatic theory of probability of finite "sample spaces".
		\\
		"Conditional probability" is the probability of "family of events" A, given that the "family of events B" occurred.\\
		$$P(A|B) \triangleq \frac{P(A \cap B)}{P(B)}$$\\
		$A$ and $B$ are "statistical independent" if:\\
		$$P(A \cap B) = P(A)*P(B)$$
		wich can be extended to:\\
		$$ P\big(\bigcap_{\forall i}{ A_i}\big) = \prod_{\forall i}{P(A_i)} $$\\
		Given the last two properties we have that the "conditional property" of two "statistical independent" "family of events" is:\\
		$$ P(A|B) = \frac{P(A \cap B)}{P(B)} = \frac{P(A)*P(B)}{P(B)} = P(A) $$.
		The "Theorem of total probability" states that:\\
		$$P(B) = \sum_{\forall i}{P(A_i|B)}$$
		
		\section{Reference}
		\begin{enumerate}
			\item {The Concise Encyclopedia of Statistics - Yadolah Dodge}
		\end{enumerate}

\end{document}