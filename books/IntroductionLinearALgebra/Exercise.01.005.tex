\documentclass[12pt]{article}
\usepackage{amsmath}
\usepackage{pdfpages}
\title{2D Graphics with Asymptote}
\author{The Asymptote Project}
\newcommand{\insertrep}[1]{%
	\hspace*{-2.4cm}
	\fbox{\includegraphics[page=1,scale=2]{#1}}
}
\begin{document}
Exercise 5:\\
Compute $u+v+w$ and $2u+2v+w$. How do you know $u$, $v$, $w$ lie in a plane.\\
$$u = \begin{bmatrix}1\\2\\3\end{bmatrix}$$
$$v = \begin{bmatrix}-3\\1\\-2\end{bmatrix}$$
$$w = \begin{bmatrix}2\\-3\\-1\end{bmatrix}$$
One should expect that three vectors must awayls express any $R^3$ point as a linear combination, but in this example we can see, the linear combination of $u+v+w$ "only" express a plane.\\
One hint is that summing all vector we have the zero vector. This means that the third vector does not increase the expressiveness of the linear combination.\\
This happen because the vector $w$ can be written as a linear combination of $v$ and $w$.\\
\\
The following proof explains why $w$ does not increase the expressiveness of $u$ and $v$.
\begin{align*}
	u+v+w=0\\
	u+v=-w\\
	-u-v=w\\
	\\
	A = au + bv + cw\\
	A = au + bv + c(-u-v)\\
	A = au + bv - cu- cv\\	
	A = (a-c)u + (b-c)v\\
	A = a_2u + b_2v
\end{align*}
\newpage
In this image, the blue plane represents all possible linear combinations of all three vectors. The red dots are random linear combinations.
\begin{center}
	\insertrep{exercise01005a.pdf}
\end{center}
\newpage
In this image we see exactly how all random linear combinations fall in the plane.
\begin{center}
	\insertrep{exercise01005b.pdf}
\end{center}
\end{document}