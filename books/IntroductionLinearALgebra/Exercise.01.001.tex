\documentclass[12pt]{article}
\usepackage{amsmath}
\usepackage{pdfpages}
\title{2D Graphics with Asymptote}
\author{The Asymptote Project}
\newcommand{\insertrep}[1]{%
	\hspace*{-2.4cm}
	\fbox{\includegraphics[page=1,scale=1]{#1}}

}
\begin{document}
Exercise 1:\\
Describe geometrically (line, plane, or all of $R^3$) all linear combinations of:\\\\
Linear combination is $a*A+b*B$ where $a \in R$ and $b \in R$.\\
\\
(a) $A = \begin{bmatrix}1\\2\\3\end{bmatrix}$ and $B = \begin{bmatrix}3\\6\\9\end{bmatrix}$\\
\\
Both points are on the same line, so their linear combination can only express a line. The blue line is the line that is formed by all linear combinations of $A$ and $B$. The red points are random linear combinations of $A$ and $B$ that we confirm that fall all in the line.\\
\begin{center}
\insertrep{exercise01001a.pdf}
\end{center}
\newpage
(b)  $A = \begin{bmatrix}1\\0\\0\end{bmatrix}$ and $B = \begin{bmatrix}0\\2\\3\end{bmatrix}$\\
It is obvious that we cannot have one line passing throught $(0,0,0)$ that also passes throught $(1,0,0)$ and $(0,2,3)$. So, the linear combinations of theses points forms a plane.\\
The blue plane represents all linear combinations of $A$ and $B$.\\
The red poins are, again, random linear combinations of $A$ and $B$. We can see in the second figure how all points fall perfectly in the plane.
\begin{center}
\insertrep{exercise01001b1.pdf}
\end{center}
\begin{center}
	\insertrep{exercise01001b2.pdf}
\end{center}
\newpage
(c) $A = \begin{bmatrix}2\\0\\0\end{bmatrix}$ and $B = \begin{bmatrix}0\\2\\2\end{bmatrix}$ and $C = \begin{bmatrix}2\\2\\3\end{bmatrix}$\\
\\
In this case it is also impossible to have a line passing through $(0,0,0)$, $(2,0,0)$ and $(0,2,2)$, but it is also impossible to have a plane passing through $(2,2,3)$. So, all linear combinations of $A$, $B$, $C$ span all possible points of $R^3$.\\
Again, we show the blue dashed box as a representation of the space, but the linear combinations are not limited inside this blue box, they will span all $R^3$.\\
Deceivingly, the random points appear to span a plane. That happens because $C$ is almost the sum of $A$ and $B$. The second picture show us that the points are not perfectedly in the plane. The third picture helps us to see the difference because it shows what happens when $C=(2,2,2)$ and the linear combination fall all in the plane.
\begin{center}
	\insertrep{exercise01001c1.pdf}
\end{center}
\begin{center}
	\insertrep{exercise01001c2.pdf}
\end{center}
\begin{center}
	\insertrep{exercise01001c3.pdf}
\end{center}
\end{document}