\documentclass[12pt]{article}
\usepackage{amsmath}
\usepackage{pdfpages}
\usepackage{lscape}
\usepackage{hyperref}
\title{2D Graphics with Asymptote}
\author{The Asymptote Project}
\newcommand{\insertrep}[1]{%
	\hspace*{-2.4cm}
	\fbox{\includegraphics[page=1,scale=1.5]{#1}}
}
\begin{document}
	Exercise 12:\\
	How many corners does a cube have in 4 dimensions? How many 3D faces? How many edges?
	
	\section{Vertices}
	A typical corner is $(0,0,1,0)$. A typical edge goes to $(0,1,0,0)$.\\
	\\
	In 3D, all cube corners are:\\
	$(0,0,0)$\\
	$(0,0,1)$\\
	$(0,1,0)$\\
	$(0,1,1)$\\
	$(1,0,0)$\\
	$(1,0,1)$\\
	$(1,1,0)$\\
	$(1,1,1)$\\
	\\
	Which follows the permutation logic, and we know that in a set with three members (3D space), each with two possibilities (0 or 1) we have $2^3=8$ permutations.\\
	
	\section{Edges}
	In a 4D space we will have $2^4=16$ permutations, so we will have 16 corners.\\
	If we use the technique of imagining a 4D-Cube as a moving 3D Cube, we will have:\\
	1 - 12 edges from the "starting" cube;\\
	2 - 12 edges from the "final" cude;\\
	3 - 8 edges from the "moving vertices"\\
	\\
	What will give us 32 edges on a 4D-Cube.
	\newpage
	The recursive formula is: $f(N) = 2*f(N-1) + 2^{N-1}$
	\\
	\begin{align*}
		f(N) = 2f(N-1)+2^{N-1}\\
		f(N+1) = 2f(N)+2^{N}\\
		f(N+1) - 2f(N) = 2^{N}\\
		(E-2)f = 2^{N}\\
		(E-2)(E-2)f = (E-2)*2^{N}\\
		(E-2)^2f = 0\\
		\\
		(\alpha n + \beta)2^n\\
		\\
		(1\alpha + \beta)2^1 = 1\\
		(2\alpha + \beta)2^2 = 4\\
		(3\alpha + \beta)2^3 = 12\\
		\\
		2\alpha + 2\beta = 1\\
		8\alpha + 4\beta = 4\\
		24\alpha + 8\beta = 12\\
		\\
		-2(2\alpha + 2\beta) = -2*1\\			
		8\alpha + 4\beta = 4\\
		\\
		-4\alpha - 4\beta = -2\\			
		8\alpha + 4\beta = 4\\
		\\
		-4\alpha - 4\beta = -2\\			
		8\alpha + 4\beta = 4\\		
		\\
		4\alpha + 0\beta = 2\\
		\\
	\end{align*}
	\begin{align*}
		4\alpha = 2\\		
		\alpha = \frac{2}{4}\\
		\alpha = \frac{1}{2}\\
		\\
		2\frac{1}{2} + 2\beta = 1\\
		\frac{2}{2} + 2\beta = 1\\		
		1 + 2\beta = 1\\
		2\beta = 1 - 1\\
		2\beta = 0\\		
		\beta = 0
		\\
		(\frac{n}{2})2^n\\		
		\frac{n*2^n}{2}\\
	\end{align*}
	Which is a "closed form" for the quantities of edges of "cube" in any dimension:
	$$ \text{edges}(n) = \frac{n*2^n}{2}$$	
	Which can be interpreted as the dimension times half the vertices. This makes total sense because at each vertex there are three edges, one for each each axis in that dimension. This counts each edge twice, at the start and at the end, so we divide by two.
	\newpage
	\\
	Reference:\\
	\\
	Beyond the Third Dimension\\	
	by \\
	Thomas F. Banchoff \\
	\begin{quote}		
	It is helpful to think of cubes as generated by lower-dimensional cubes in motion. A point in motion generates a segment; a segment in motion generates a square; a square in motion generates a cube; and so on. From this progression, a pattern develops, which we can exploit to predict the numbers of vertices and edges.\\
	Each time we move a cube to generate a cube in the next higher dimension, the number of vertices doubles. That is easy to see since we have an initial position and a final position, each with the same number of vertices. Using this information we can infer an explicit formula for the number of vertices of a cube in any dimension, namely 2 raised to that power.\\
	What about the number of edges? A square has 4 edges, and as it moves from one position to the other, each of its 4 vertices traces out an edge. Thus we have 4 edges on the initial square, 4 on the final square, and 4 traced out by the moving vertices for a total of 12. That basic pattern repeats itself. If we move a figure in a straight line, then the number of edges in the new figure is twice the original number of edges plus the number of moving vertices. Thus the number of edges in a four-cube is 2 times 12 plus 8 for a total of 32. Similarly we find 32 + 32 + 16 = 80 edges on a five-cube and 80 + 80 + 32 = 192 edges on a six-cube.\\
	This presentation definitely suggests a pattern, namely that the number of edges of a hypercube of a given dimension is the dimension multiplied by half the number of vertices in that dimension. Once we notice a pattern like this, it can be proved to hold in all dimensions by mathematical induction.
	\end{quote}
	\\
	\url{http://www.math.brown.edu/~banchoff/Beyond3d/chapter4/section05.html}
	\\
	\section{Faces}
	The easiest way to calculate the quantity of faces of a 4D cube is to realize that from each vertex starts four edges and each face is a combination of two of theses edges. This gives us ${4 \choose 2}=6$ faces per vertex.\\
	A 4D Cube has $2^4=16$ vertices each containing 6 faces. But this algotihm counts each face 4 times, so we actually have:
	$$f(n) = \frac{2^n*{n \choose 2}}{4}$$
	Which gives us the correct value for $R^4$, $\frac{2^4*{4 \choose 2}}{4}=24$.\\
	\\	
	Reference:\\
	\begin{quote}
		The hypercube is so highly symmetric that every vertex looks like every other vertex. If we know what happens at one vertex, we can figure out what is going to happen at all vertices. At each vertex there are as many square faces as there are ways to choose 2 edges from among the 4 edges at the point, namely 6. Since there are 16 vertices, we can multiply 6 by 16 to get 96, but this counts each square four times, once for each of its vertices. The correct number of squares in a hypercube is then 96/4, or 24.
	\end{quote}
	http://www.math.brown.edu/~banchoff/Beyond3d/chapter4/section07.html
\end{document}