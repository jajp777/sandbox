\documentclass{article}
\usepackage[latin1]{inputenc}
\usepackage{amsmath}
\usepackage{amsfonts}
\usepackage{amssymb}
\usepackage{graphicx}
\usepackage{hyperref}
\usepackage{pgfplots}
\begin{document}
	Exercise 1.5.13\\
	In the previous exercise, the matrix $A^{-1}$ is full; none of its entries are zero.\\ 
	(a) What is the physical significance of this fact? (Think of the equation $x = A^{-1}b$, 
	especially in the case where only one entry, say the jth, of b is nonzero. If the 
	(i,j) entry of $A^{-1}$ were zero, what would this imply? Does this make physical 
	sense?)\\ 
	(b) The entries of $A^{-1}$ decrease in magnitude as we move away from the main 
	diagonal? What does this mean physically?\\	
	\\
	Answers:\\
	(a) $x = A^{-1}b$ means that $x_{i}$ is the inner product $\text{row}_i(A).b_i$ which means that the i-th displacement is the linear combination of a non-zeros coefficients and the applied forces, which means that every displacement influence every cart.\\
	In this special case that b is zero in all but one place, we would have only one force applied, and this force will cause a particular displacement in each cart.\\
	If for some reason the (in) entry of A were zero, that would mean that somehow a force applied to the j-th cart would not affect the i-th spring. Maybe they are not connected, so the energy apllied could not flow to that spring. One can easily see this imagining a extra, and disconnected, spring that will have its "influence factors" all zeroed beecaused it does not makes "part" of the system.\\
	\\
	(b) Yes! And this means that the displacement at the i-th cart will be more affected by the i-th force.
\end{document}