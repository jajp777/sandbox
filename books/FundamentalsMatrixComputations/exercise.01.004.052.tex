\documentclass{article}
\usepackage[latin1]{inputenc}
\usepackage{amsmath}
\usepackage{amsfonts}
\usepackage{amssymb}
\usepackage{graphicx}
\usepackage{hyperref}
\begin{document}
	Proposition 1.4.51\\
	If A is positive definite, then $a_{ii} > 0$ for $i = 1,..., n$. 
	\\
	Exercise 1.4.52\\
	Prove Proposition 1.4.51. Do not use the Cholesky decomposition in your 
	proof; we want to use this result to prove that the Cholesky decomposition exists.\\ 
	(Hint: Find a nonzero vector $x$ such that $x^TAx = a_{ii}$.)\\ 
	\\
	Proof:\\
	If $A$ is Positive Definite, then $x^TAx > 0$ for all $x$.
	\begin{align*}
		x^TAx = \begin{Bmatrix}x_1&x_2&...&x_n\end{Bmatrix}*\begin{Bmatrix}
		a_{11} & ... & a{1n}\\
		... & & ...\\
		a_{n1} & ... & a{nn}		
		\end{Bmatrix}*\begin{Bmatrix}x_1\\x_2\\...\\x_n\end{Bmatrix}
	\end{align*}
	But we know that "line vector and matrix multipliction" when the "line vector" is $e_i$ just selects one of the rows of the matrix.\\
	For example
	\begin{align*}
		e_1 = \begin{Bmatrix}1\\0\\...\\0\end{Bmatrix}\\
		e_1^T*\begin{Bmatrix}
		a_{11} & ... & a_{1n}\\
		... & & ...\\
		a_{n1} & ... & a_{nn}		
		\end{Bmatrix} = \begin{Bmatrix}a_{11} & ... & a_{1n} \end{Bmatrix}
	\end{align*}
	On the other hand, "matrix and column vector multiplication", when the "column vector" is $e_i$ we choose a column of $A$;
	\begin{align*}
	e_1 = \begin{Bmatrix}1\\0\\...\\0\end{Bmatrix}\\
	\begin{Bmatrix}
	a_{11} & ... & a_{1n}\\
	... & & ...\\
	a_{n1} & ... & a_{nn}		
	\end{Bmatrix}*e_i = \begin{Bmatrix}a_{11} \\ ... \\ a_{n1} \end{Bmatrix}
	\end{align*}
	So we can "choose" a element of the matrix by doing:
	\begin{align*}
		e_i^TAe_j = a_{ij}
	\end{align*}
	So we can easily "choose" the diagonal elements by doing:
	\begin{align*}
	e_i^TAe_i = a_{ii}
	\end{align*}
	But, given that A is "Positive Definite" this expression is always positive.
	So we have:\\
	\begin{align*}
	e_i^TAe_i > 0\\
	a_{ii} > 0\\
	\square	
	\end{align*}
\end{document}