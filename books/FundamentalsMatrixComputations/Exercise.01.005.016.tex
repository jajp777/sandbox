\documentclass{article}
\usepackage[latin1]{inputenc}
\usepackage{amsmath}
\usepackage{amsfonts}
\usepackage{amssymb}
\usepackage{graphicx}
\usepackage{hyperref}
\usepackage{pgfplots}
\begin{document}
	Exercise 1.5.16\\
	(d)\\
	\\
	Think about how your subroutines could be used to calculate the inverse of a 
	positive definite matrix. Calculate $A^{-1}$, where 
	\begin{align*}
	A = \begin{bmatrix}
	1 & 1/2 & 1/3 \\
	1/2 & 1/3 & 1/4\\ 
	1/3 & 1/4 & 1/5\\ 
	\end{bmatrix}	
	\end{align*}
	It turns out that the entries of $A^{-1}$ are all integers. Notice that your computed 
	solution suffers from significant roundoff errors. This is because A is (mildly) 
	ill conditioned. This is the 3x3 member of a famous family of ill-conditioned 
	matrices called Hilbert matrices; the condition gets rapidly worse as the size 
	of the matrix increases. We will discuss ill-conditioned matrices in Chapter 2.
	\\
	Answers\\
	\\
	We can calculate a inverse matrix using Cholesky factorization because:
	\begin{align*}
		A &= R^TR\\
		A*A^{-1} &= R^TR*A^{-1}\\
		I &= R^TR*A^{-1}\\
		\begin{bmatrix}
			e_1\\e_2\\\vdots\\e_n
		\end{bmatrix} &= R^TR*A^{-1}\\
		\begin{bmatrix}
		e_1\\e_2\\\vdots\\e_n
		\end{bmatrix} &= R^TR*\begin{bmatrix}
			a^{-1}_{1} &
			a^{-1}_{2} &
			... &
			a^{-1}_{n}
		\end{bmatrix}
	\end{align*}	
	Which can be simplified as:
	\begin{align*}
		\begin{bmatrix}
			e_1 = R^TR*a^{-1}_{1}\\
			e_2 = R^TR*a^{-1}_{2}\\
			\vdots \\
			e_n = R^TR*a^{-1}_{n}
		\end{bmatrix}\\		
	\end{align*}
	In this way we can find the inverse of a Matrix with $n$ "solves" using Cholesky. Each Cholesky solve makes one backward substitution and one foward substitution. They have the same complexity: $n^2$. Which gives us a solution in $n^3$.	
\end{document}