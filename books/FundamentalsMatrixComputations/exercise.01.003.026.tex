\documentclass{article}
\usepackage[latin1]{inputenc}
\usepackage{amsmath}
\usepackage{amsfonts}
\usepackage{amssymb}
\usepackage{graphicx}
\usepackage{xcolor}
\begin{document}
	In this exercise you will use mathematical induction to draw the same\\
	conclusion as in the previous exercise. If you are weak on mathematical induction,\\ 
	you should definitely work this exercise.
	
	We wish to prove that\\
	$$\sum_{i=1}^{n-1} = \frac{n(n-1)}{2}$$
	\\
	for all positive integers n.\\
	\\
	Begin by showing that (1.3.27) holds when $n = 1$. The sum on the left-hand side is empty in this case. If you feel nervous about this, you can check the case $n = 2$ as well. Next show that if A.3.27) holds for $n = k$, then it holds also holds for $n ? k + 1$. That is, show that 
	$$\sum_{i=1}^{k} = \frac{(n+1)k}{2}$$
	is true, assuming that\\
	$$\sum_{i=1}^{k-1} = \frac{k(k-1)}{2}$$
	is true. This is just a matter of simple algebra. Once you have done this, you will 
	have proved by induction that A.3.27) holds for all positive integers n.\\
	Proof\\
	\\
	Step 1 (Base Case):
	\begin{align*}
		n = 1\\
		\sum_{i=1}^{1-1} &= \sum_{i=1}^{0} = 0\\
		\sum_{i=1}^{n-1} &= \frac{n(n-1)}{2}\\
		&= \frac{1(1-1)}{2}\\
		&= \frac{1(0)}{2}\\
		&= \frac{0}{2}\\
		&= 0\\
		\square
	\end{align*}
	\\
	Step 2\\
	\begin{align*}
		n &= n + 1\\
		\sum_{i=1}^{n}{i} &= \frac{(n+1)((n+1)-1)}{2}\\
		&= \frac{(n+1)(n+1-1)}{2}\\
		&= \frac{(n+1)n}{2}\\
		&= \frac{n^2+n}{2}\\
		&= \frac{n^2+n-n}{2}+\frac{n}{2}\\
		&= \frac{n(n-1)+n}{2}+\frac{n}{2}\\
		&= \frac{n(n-1)}{2}+\frac{n}{2}+\frac{n}{2}\\
		&= \frac{n(n-1)}{2}+\frac{2n}{2}\\
		&= \frac{n(n-1)}{2}+n\\
		&= (\sum_{i=1}^{n-1}{i})+n\\
		&= \sum_{i=1}^{n}{i}\\		
	\end{align*}
\end{document}