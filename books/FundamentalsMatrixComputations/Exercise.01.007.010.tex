\documentclass[10pt,a4paper]{article}
\usepackage[latin1]{inputenc}
\usepackage{amsmath}
\usepackage{amsfonts}
\usepackage{amssymb}
\usepackage{graphicx}
\usepackage{hyperref}
\author{Daniel Frederico Lins Leite}
\begin{document}	
	Exercise 1.7.10\\
	Let\\
	\begin{align*}
		A = \begin{bmatrix}
		2&1&-1&3\\-2&0&0&0\\4&1&-2&6\\-6&-1&2&-3
		\end{bmatrix}
	\end{align*}	
	\\
	(a) Calculate the appropriate (four) determinants to show that A can be transformed to (nonsingular) upper-triangular form by operations of type 1 only. (By the way, this is strictly an academic exercise. In practice one never calculates these determinants in advance.)\\ 
	\\
	(b) Carry out the row operations of type 1 to transform the system $Ax = b$ to an equivalent system Ux = y, where U is upper triangular. Save the multipliers 
	for use in Exercise 1.7.18.\\
	\\
	(c) Carry out the back substitution on the system $Ux = y$ to obtain the solution of $Ax = b$. Don't forget to check your work.\\
	\\
	Answers:\\
	\\
	(a)
	\begin{verbatim}
	octave:1> A = [2,1,-1,3;-2,0,0,0;4,1,-2,6;-6,-1,2,-3]
	A =	
	 2   1  -1   3
	-2   0   0   0
	 4   1  -2   6
	-6  -1   2  -3	
	octave:2> det(A(1:1,1:1))
	ans =  2
	octave:3> det(A(1:2,1:2))
	ans =  2
	octave:4> det(A(1:3,1:3))
	ans = -2
	octave:5> det(A(1:4,1:4))
	ans = -6
	\end{verbatim}
	Given that all determinants are non-zero, all sub-matrices are non-singular.\\
	\\ \\ \\ \\ \\ \\ \\ \\ \\ \\ \\
	(b)
	\\
	\begin{verbatim}
	octave:1> A = [2,1,-1,3;-2,0,0,0;4,1,-2,6;-6,-1,2,-3]
	A =
	
	 2   1  -1   3
	-2   0   0   0
	 4   1  -2   6
	-6  -1   2  -3	
		
	octave:2> b = [13;-2;24;-14]
	b =
	
	13
	-2
	24
	-14
	
	octave:3> AUG=[A,b]
	AUG =
	
	 2    1   -1    3   13
	-2    0    0    0   -2
	 4    1   -2    6   24
	-6   -1    2   -3  -14
	
	octave:4> L1 = [0,0,0,0,0;-1,0,0,0,0;2,0,0,0,0;-3,0,0,0,0]
	L1 =
	
	 0   0   0   0   0
	-1   0   0   0   0
	 2   0   0   0   0
	-3   0   0   0   0

	octave:5> AUG1 = AUG - L1*AUG
	AUG1 =
	
	2    1   -1    3   13
	0    1   -1    3   11
	0   -1    0    0   -2
	0    2   -1    6   25
	
	octave:6> L2 = [0,0,0,0;0,0,0,0;0,-1,0,0;0,2,0,0]
	L2 =
	
	0   0   0   0
	0   0   0   0
	0  -1   0   0
	0   2   0   0
	
	octave:7> AUG2 = AUG1 - L2*AUG1
	AUG2 =
	
	2    1   -1    3   13
	0    1   -1    3   11
	0    0   -1    3    9
	0    0    1    0    3
	
	octave:8> L3 = [0,0,0,0;0,0,0,0;0,0,0,0;0,0,-1,0]
	L3 =
	
	0   0   0   0
	0   0   0   0
	0   0   0   0
	0   0  -1   0
	
	octave:9> AUG3 = AUG2 - L3*AUG2
	AUG3 =
	
	2    1   -1    3   13
	0    1   -1    3   11
	0    0   -1    3    9
	0    0    0    3   12
	\end{verbatim}
	
	(c)
	
	\begin{verbatim}	
	octave:10> AUG3(1:4,1:4)\AUG3(1:4,5:5)
	ans =
	
	1
	2
	3
	4
	\end{verbatim}
	
	Checking:
	
	\begin{verbatim}
	octave:11> A\b
	ans =
	
	1.0000
	2.0000
	3.0000
	4.0000	
	\end{verbatim}
\end{document}
