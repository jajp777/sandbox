\documentclass{article}
\usepackage[latin1]{inputenc}
\usepackage{amsmath}
\usepackage{amsfonts}
\usepackage{amssymb}
\usepackage{graphicx}
\usepackage{hyperref}
\begin{document}
	Theorem 1.4.4\\
	Let $M$ be any $n\times n$ nonsingular matrix, and let $A = M^TM$. Then 
	$A$ is positive definite.\\
	\\	
	Proposition 1.4.55\\
	If $A$ and $X$ are $n \times n$, A is "Positive Definite", and $X$ is nonsingular 
	then the matrix $B = X^TAX$ is also "Positive Definite".\\ 
	Considering the special case $A = I$ (which is clearly "Positive Definite"), we see that 
	this proposition is a generalization of Theorem 1.4.4.\\ 
	\\
	Exercise 1.4.56\\
	Prove Proposition 1.4.55.\\
	\\
	Proof:\\
	\\
	\begin{align*}
		z^TAz > 0 & \text { because A is "PD"}\\
		\\
		Xy=z\\
		(Xy)^TA(Xy)>0\\
		y^TX^TAXy>0\\
		y^TBy>0 & \text{ if } y > 0
	\end{align*}
	We already know that $z>0$ because of the "Positive Difiniteness" of A. So we need:
	\begin{align*}
		Xy=z\\
		z>0\\
		y>0\\
		Xy>0
	\end{align*}
	But if X is "Singular" it is possible that a $y>0$ that $Xy=0$. So to guarantee that $y>0$ and $Xy>0$ we need that $X$ be "NonSingular".
\end{document}