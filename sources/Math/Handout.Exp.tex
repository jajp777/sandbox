\documentclass{article}
\usepackage[utf8]{inputenc}
\usepackage[english]{babel} 
\usepackage{amsmath,amssymb,amsthm}
\newtheorem{theorem}{Theorem}[section]
\newtheorem{corollary}{Corollary}[theorem]
\newtheorem{lemma}[theorem]{Lemma}
\title{Mathematics Handout - Logaritmics}
\author{Daniel Frederico Lins Leite}
\date{July 2016}
\begin{document}
\section{Introduction}
\maketitle
\begin{proof}{Exp is continuous}
\[x^l = A => log_x A = l\]
\[x^m = B => log_x B = m\]      
\[x^n = A*B => log_x(A*B) = n\]
\[x^n = x^l*x^m\]
\[x^n = x^(l+m)\]
Given that power in bijective and onto *
\[n = l+m\]
\[log_x (A*B) = log_x A + log_x B\]
\end{proof}
\begin{proof}{Logarithm Division Rule}
\[x^l = A => log_x A = l\]
\[x^m = B => log_x B = m\]      
\[x^n = A/B => log_x(A/B) = n\]
\[x^n = (x^l)/(x^m)\]
\[x^n = x^l*x^(-m)\]
Given that power in bijective and onto *
\[n=l-m\]
\[log_x (A/B) = log_x A - log_x B\]
\end{proof}
\begin{proof}{Logarithm Division Rule}
\[log_x A = B => x^B = A\]
First multiply the left side for c
\[C*(log_x A) = C*B\]
Now  raise the right side to C
\[(x^B)^C =  A^C\]
Given the Power rule
\[x^(B*C) = A^C\]
We can write this expression as log as
\[log_x (A^C) = B*C\]
Which give use
\[log_x (A^C) = C*(log_x A)\]
\end{proof}
\end{document}