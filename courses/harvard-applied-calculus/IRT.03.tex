\documentclass[10pt,a4paper]{article}
\usepackage[latin1]{inputenc}
\usepackage{amsmath}
\usepackage{amsfonts}
\usepackage{amssymb}
\usepackage{graphicx}
\author{Daniel Frederico Lins Leite}
\begin{document}
	Question 03:
	Which piece(s) of information guarantee that the graph of P is always increasing?\\
	Answer:
	\begin{align*}
		\frac{d}{d\theta} {(1-c)} = 0\\
		\\
		{1+e^{-a(\theta-b)}} 
		&= {1+e^{-a\theta+ab}} \\
		&= {1+e^{-a\theta}e^{ab}} \\
		\frac{d}{d\theta} \big[ {1+e^{-a\theta}e^{ab}} \big] &=
		e^{ab}*-a*e^{-a\theta}\\
		&= -ae^{ab}e^{-a\theta}\\
		&= -ae^{a(b-\theta)}\\		
		&= -ae^{-a(\theta-b)}\\				
		\\
		(1+e^{-a(\theta-b)})^2 &= (1^2 + 2*1*e^{-a\theta}e^{ab} + (e^{-a\theta}e^{ab})^2)\\
		&= (1 + 2e^{-a\theta}e^{ab} + e^{-2a\theta}e^{ab})\\
		\\
		\frac{d}{d\theta} \big[ {c+\frac{1-c}{1+e^{-a(\theta-b)}}} \big] =\\
		&= 0+\frac{d}{d\theta} \big[ {\frac{1-c}{1+e^{-a(\theta-b)}}} \big] =\\
		&= 0+\frac{d}{d\theta} \big[ {\frac{1-c}{1+e^{-a(\theta-b)}}} \big] =\\
		&= \frac{ -(1-c)(ae^{-a(\theta-b)}) + 0*(...) } {(1+e^{-a(\theta-b)})^2}\\		
		&= \frac{ -(1-c)(-ae^{-a(\theta-b)})} {(1+e^{-a(\theta-b)})^2}\\
		&= \frac{ a(1-c)e^{-a(\theta-b)}} {(1+e^{-a(\theta-b)})^2}\\
		\\
		P'(b) &= \frac{ a(1-c)e^{-a(b-b)}} {(1+e^{-a(b-b)})^2}\\		
		= &\frac{ a(1-c)e^{-a(0)}} {(1+e^{-a(0)})^2}\\
		= &\frac{ a(1-c)e^{0}} {(1+e^{0})^2}\\		
		= &\frac{ a(1-c)} {(1+1)^2}\\				
		= &\frac{ a(1-c)} {4}\\						
	\end{align*}
\end{document}